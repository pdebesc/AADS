\section*{Exercise 26.3-2}
Proof of Theorem 26.10.
\\
\\
The proof is by induction. 

\emph{The base case, the first iteration}

First, the Ford-Fulkerson method initializes the flow to 0 on all edges. Hence at iteration 0, no iteration have run, all flow values are 0 and the outgoing flow is 0. Hence the induction base case is valid.

Now assume that for iteration $n$ the theorems statement is true, we now want to show that for iteration $n+1$ the statement is also true.

At the beginning of iteration $n+1$ two things can happen.

1) There is no path from $s$ to $t$ in $G_f$.

If there is no path, then the Ford-Fulkerson method finishes with the values found from iteration $n$. Those values are true by the induction step assumptions.

2) There is a path from $s$ to $t$ in $G_f$.

In this case the algorithm determines $c_f(p)$ which is an integer, since that value is chosen from the integer residual capacities that is the outcome of the  $n$'th iteration. All existing flow on path $p$ are updated with the $c_f(p)$. The existing flow is integer because of the induction assumption and adding or subtracting the integer $c_f(p)$ will yield an integer. Since the flow $f$ is only updated with the integer capacity $c_f(p)$, the new $\abs{f}$ that yields from the iteration will also only be updated with that integer value. Thus $\abs{f}$ is also integer after iteration $n+1$.