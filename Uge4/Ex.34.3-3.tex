\section*{Ex.34.3-3}
\subsection*{ $\bar{L}\leq_P L \Leftrightarrow L \leq_P \bar{L}$}

We start by showing $\Rightarrow$:
\\
So assume $\bar{L}\leq_P L$.

We want to show that for every $x\in \bar{L} \Leftrightarrow  f(x) \in L$

For all $x\in \bar{L}$ it holds that
\begin{align*}
x\in \bar{L} & \Leftrightarrow x \not\in L 					&\text{as $L$ and $\bar{L}$ are complements}\\
						 & \Leftrightarrow f(x) \not\in \bar{L} &\text{as $\leq_P$ implies a reducible function exists}\\
						 & \Leftrightarrow f(x) \in L 					&\text{as $L$ and $\bar{L}$ are complements}
\end{align*}
Which show the first part.

Now we show $\Leftarrow$:
\\
Assume $L \leq_P \bar{L}$.

We want to show that for every $x\in L \Leftrightarrow  f(x) \in \bar{L}$

For all $x\in L$ it holds that
\begin{align*}
x\in L       & \Leftrightarrow x \not\in \bar{L} 					&\text{as $L$ and $\bar{L}$ are complements}\\
						 & \Leftrightarrow f(x) \not\in L  &\text{as $\leq_P$ implies a reducible function exists}\\
						 & \Leftrightarrow f(x) \in \bar{L} 					&\text{as $L$ and $\bar{L}$ are complements}
\end{align*}
Which show the other arrow in the bi--implication. Hence we have shown what we wanted.