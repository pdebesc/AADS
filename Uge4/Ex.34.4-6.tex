\section*{Ex.34.4-6}
\subsection*{Formula satisfiability}

We are given an algorithm, call it $A$, which can decide formula satisfiability in polynomial time. To find a satisfying assignment with the use of $A$ do the following:

Let $x_1,\ldots,x_k$ be the $k$ input variables to the formula problem. First call $A$, to decide if the formula instance has a satisfying assignment. If not it will return 0, and no more work needs to be done. If it returns 1, there exists an assignment which we can find.

Note that if we set $x_i=1$ for some $i\in\{1,\ldots,k\}$, and then run $A$ it will tell us if there exists a solution when $x_i=1$. If there doesn't, then set $x_1 =0$. Do this for all $i\in\{1,\ldots,k\}$, and you will get a satisfying assignment. For each variable we call $A$ once, which is $O(n^m)$, and we do this $k$ times plus some constant work. This yields an algorithm that finds the satisfying assignment in $O(kn^m)=O(n^m)$ time. Which is polynomial.

