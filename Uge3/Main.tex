\RequirePackage[l2tabu, orthodox]{nag}
\documentclass[a4paper,extrafontsizes,12pt,final,article]{memoir} % DOKUMENTKLASSE. Mulighederne er bl.a. article, report, book, memoir     %
%%%%%%%%%%%%%%%%%%%%%%%%%%%%%%%%%%%%%%%%%%%%%%%%%%%%%%%%%%%%%%%%%%%%%%%%%%%%%%%%%%%%%%%%%%%%%%%%%%%%%%%%%%%%%%%%%%%%%%%%%%%%%%%%%%%%%%%%%%%%%%%
%%%%%%%%%%%%%%%%%%%%%%%%%%%%%%%%%%%%%%%%%%%%%%          PAKKER DER INDLÆSES         %%%%%%%%%%%%%%%%%%%%%%%%%%%%%%%%%%%%%%%%%%%%%%%%%%%%%%%%%%%
%%%%%%%%%%%%%%%%%%%%%%%%%%%%%%%%%%%%%%%%%%%%%%%%%%%%%%%%%%%%%%%%%%%%%%%%%%%%%%%%%%%%%%%%%%%%%%%%%%%%%%%%%%%%%%%%%%%%%%%%%%%%%%%%%%%%%%%%%%%%%%%
\usepackage[T1]{fontenc}
\usepackage[utf8]{inputenc}                      % Lidt kodning så der ikke kommer problemer ved visse konverteringer                      %
\usepackage[english]{babel}                          % Dansk sprogpakke                                                                        %
\usepackage{amssymb,amsmath,bm,mathtools}           % Matematiske tegn og skrifttyper, bl.a. \mathbb{}                                        %
\usepackage{enumitem}
\usepackage{calc}                                   % giver \setlength, og andre muligheder                                                   %
\usepackage{hyperref}                               % Styre referencer                                                                        %
\usepackage[amsmath,thmmarks,amsthm]{ntheorem}		% Sætningskonstruktioner																											%
\usepackage{memhfixc}                               % fix til memoir, så hyperref virker                                                      %
\usepackage{placeins}                               % Giver \FloatBarrier                                                                     %
\usepackage{flafter}                                % Styre at float ikke kommer for tideligt (i sectionen før)                               %
\usepackage{fancyvrb}																% Verbatim pakke
\usepackage[version=3]{mhchem}
\usepackage[obeyDraft]{todonotes}														% giver \todo{}
\usepackage{microtype}
\usepackage{nth}																	% brug \nth{tal} giver tal'nth
\usepackage{xfrac}																% Giver \sfrac
\mathtoolsset{showonlyrefs}													%	 Vis kun equation der er har en \eqref henvisning
\usepackage{float} %Man kan bruge H til at sætte en table eller figure netop hvor den er i koden
\usepackage{subfig} %Man kan sætte billeder ved siden af hinanden med \subfloat[overskrift]{indsæt billede}
%%%%%%%%%%%%%%%%%%%%%%%%%%%%%%%%%%%%%%%%%%%%%%%%%%%%%%%%%%%%%%%%%%%%%%%%%%%%%%%%%%%%%%%%%%%%%%%%%%%%%%%%%%%%%%%%%%%%%%%%%%%%%%%%%%%%%%%%%%%%%%%
%%%%%%%%%%%%%%%%%%%%%%%%%%%%%%%%%%%%%%%%%%%%             Tekstblok            %%%%%%%%%%%%%%%%%%%%%%%%%%%%%%%%%%%%%%%%%%%%%%%%%%%%%%%%%%%%%%%%%
%%%%%%%%%%%%%%%%%%%%%%%%%%%%%%%%%%%%%%%%%%%%%%%%%%%%%%%%%%%%%%%%%%%%%%%%%%%%%%%%%%%%%%%%%%%%%%%%%%%%%%%%%%%%%%%%%%%%%%%%%%%%%%%%%%%%%%%%%%%%%%%
\setlxvchars[\normalfont]                           % Beregner bredden 65 tegn i nuværende font                                               %
\settypeblocksize{*}{1.15\lxvchars}{1.61803}        % Specifikation af tekstblok                                                              %
\setlrmargins{*}{*}{1}                              % Højre- venste margin                                                                    %
\setulmargins{*}{*}{1}                              % Op- ned margin                                                                          %
\checkandfixthelayout[nearest]                      % Kontrolere layoutet                                                                     %
%\linespread{1.3}
\nonzeroparskip																			% gives new line with paragraph change
\setlength{\parindent}{0pt}													% removes paragraphs indent
%%%%%%%%%%%%%%%%%%%%%%%%%%%%%%%%%%%%%%%%%%%%%%%%%%%%%%%%%%%%%%%%%%%%%%%%%%%%%%%%%%%%%%%%%%%%%%%%%%%%%%%%%%%%%%%%%%%%%%%%%%%%%%%%%%%%%%%%%%%%%%%
%%%%%%%%%%%%%%%%%%%%%%%%%%%%%%%%%%%%%%%%%%%%             Pagestyle            %%%%%%%%%%%%%%%%%%%%%%%%%%%%%%%%%%%%%%%%%%%%%%%%%%%%%%%%%%%%%%%%%
%%%%%%%%%%%%%%%%%%%%%%%%%%%%%%%%%%%%%%%%%%%%%%%%%%%%%%%%%%%%%%%%%%%%%%%%%%%%%%%%%%%%%%%%%%%%%%%%%%%%%%%%%%%%%%%%%%%%%%%%%%%%%%%%%%%%%%%%%%%%%%%
\pagestyle{plain}
%%%%%%%%%%%%%%%%%%%%%%%%%%%%%%%%%%%%%%%%%%%%%%%%%%%%%%%%%%%%%%%%%%%%%%%%%%%%%%%%%%%%%%%%%%%%%%%%%%%%%%%%%%%%%%%%%%%%%%%%%%%%%%%%%%%%%%%%%%%%%%%
%%%%%%%%%%%%%%%%%%%%%%%%%%%%%%%%%%%%%%%%%%%%%%%%%%%%%%%%%%%%%%%%%%%%%%%%%%%%%%%%%%%%%%%%%%%%%%%%%%%%%%%%%%%%%%%%%%%%%%%%%%%%%%%%%%%%%%%%%%%%%%%
%%%%%%%%%%%%%%%%%%%%%%%%%%%%%%%%%%%%%%%%%%%%%%%%%%    Egne makroer    %%%%%%%%%%%%%%%%%%%%%%%%%%%%%%%%%%%%%%%%%%%%%%%%%%%%%%%%%%%%%%%%%%%%%%%%%
%%%%%%%%%%%%%%%%%%%%%%%%%%%%%%%%%%%%%%%%%%%%%%%%%%%%%%%%%%%%%%%%%%%%%%%%%%%%%%%%%%%%%%%%%%%%%%%%%%%%%%%%%%%%%%%%%%%%%%%%%%%%%%%%%%%%%%%%%%%%%%%
\DeclarePairedDelimiter{\abs}{\lvert}{\rvert} % Numerisk værdi
\DeclareMathOperator*{\argmin}{\arg\,\min}
\DeclareMathOperator*{\argmax}{\arg\,\max}
\newcommand{\N}{\mathbb N} %the natural numbers
\newcommand{\Z}{\mathbb Z} %the integers
\newcommand{\Q}{\mathbb Q} %the rational numbers
\newcommand{\R}{\mathbb R} %the real numbers
\newcommand{\C}{\mathbb C} %the complex numbers
\newcommand{\Pow}{\mathbb P} %the Power set
\newcommand{\E}{\textrm{E}} %the expected value
\newcommand{\V}{\textrm{V}} %the variance 
\newcommand{\T}{\mathsf{T}} %the transpose sign
% Modulo, command \imod
\makeatletter                                       % Laver modolus
\def\imod#1{\allowbreak\mkern10mu({\operator@font mod}\,\,#1)}
\makeatother
%
\newcommand*\uafh{{\perp\mskip-11mu\perp}} % Uafhængighedssymbol
\renewcommand{\epsilon}{\varepsilon} %so that \epsilon makes a proper epsilon
\renewcommand{\phi}{\varphi} %so that \phi makes a proper phi
%\renewcommand\danishhyphenmins{22}
\addto\captionsdanish{\renewcommand{\contentsname}{Indholdsfortegnelse}} % Ændre indholdsfortegneslen i den danske sprogpakke i babel.
%%%%%%%%%%%%%%%%%%%%%%%%%%%%%%%%%%%%%%%%%    Sætningskonstruktioner     %%%%%%%%%%%%%%%%%%%%%%%%%%%%%%%%%%%%%%%%%%%%%%%%
%%%%%%%%%%%%%%%%%%%%%%%%%%%%%%%%%%%%%%%%%%%%%%%%%%%%%%%%%%%%%%%%%%%%%%%%%%%%%%%%%%%%%%%%%%%%%%%%%%%%%%%%%%%%%%%%%%%%%%%%%%%%%%%%%%%%%%%%%%%%%%%
\counterwithout{section}{chapter}

%\counterwithout{subsection}{chapter}
%\counterwithout{subsection}{section}
\setsecnumdepth{subsection}
\settocdepth{subsection}
%\setcounter{tocdepth}{2} % Dybde af indholdsfortegnelsen, 1 resulterer i at kun sections kommer med, 2 er section og subsec. etc.             %
%\setcounter{secnumdepth}{5}

%%%%%%%%%%%%%%%%%%%%%% DRAFT OPTIONS %%%%%%%%%
%\linespread{1.1} % double linespread
%\OnehalfSpacing
%\DoubleSpacing




\begin{document}
\section*{Ex.2.1}
\subsection*{Is the truly independent hash function universial}

Yes - the truly independent hash function universial. 
To prove it we want to show that $P(h(x) = h(y))\leq 1/m$. 

The truely random hash function $h$ is an independent and identical discrete uniformly distributed random variable. 
Hence $h(x)\sim U(0,m-1)$ which means $P(h(x) = j) = 1/m$ for all $j\in[m]$. So for two distinct keys $x,y \in U$
\begin{align*}
P(h(x) = h(y)) &= \sum_{j=0}^{m-1} P([h(x)=j] \cap [h(y)=j])&\\
               &= \sum_{j=0}^{m-1}   P(h(x)=j) P(h(y)=j)    &\text{equality holds as $h$ is independent}\\
							 &= m P^2(h(x)=j)                             &\text{equality holds as $h$ is identical distributed}\\
							 &= m \left(\frac{1}{m}\right)^2              &\text{equality holds as $h$ is uniform distributed}\\
							 &= \frac{1}{m}                               &
\end{align*}
Which shows that the truly independent hash function universial.
\section*{Ex.2.2}
\subsection*{Size of hash function $h$ output for collision probability is 0}



\section*{Ex.2.3}
\subsection*{Is the identity function $f$ a universal hash function?}
No $f$ is not a universal hash function. By the definition of a universal hash function, $h$ must be a random variable with some distribution, which the identity function is not.

\section*{Ex.2.4}
\subsection*{What is the expected number of elements in $L[h(x)]$ if $x \in S$}

The expected number of elements is
\begin{align*}
E\left[ \abs*{L[(h(x))]} \right] 
       &= \sum_{y\in S} P(h(y) = h(x)) \\
       &= 1 + \sum_{y\in S\backslash\{x\}} P(h(y) = h(x)) \\
       &\leq 1 + \frac{n-1}{m} < 2
\end{align*}
The first equality is the same as in the notes. The next equality we used the assumption that $x\in S$ which will give a give a collision with probability 1. Hence that element is extracted. The last equality comes from the fact that $h$ is universal hence each collision probability is less than $1/m$ and there are $n-1$ elements in the sum.

\emph{What is the bound if $h$ is 2 universal?}
\\
The calculations would be the same, the only thing that will change is where the inequality sign is. For 2 universal the probability is $P(h(x)=h(y))\leq \frac{2}{m}$. There are $n-1$ of those probabilities in the sum, hence the result would be
$$
E\left[ \abs*{L[(h(x))]} \right]  \leq 1 + \frac{2(n-1)}{m}
$$
\section*{Ex.2.5}
\subsection*{what is the probability that there is a key $y \in S$ such that $h(y) = h(x)$ and $s(y) = s(x)$}
The set $(h(x)=h(y))\cap(s(x)=s(y))$ is the set where the hash function and the signature function gives the same value for two distinct keys $x,y$. Hence probability we want to determine is the probability for that set over all possible $y$ values in $S$. So we look at
\begin{align*}
\sum_{y\in S}P[(h(x)=h(y))\cap(s(x)=s(y))] 
             &= \sum_{y\in S}P(h(x)=h(y))P(s(x)=s(y))\\
             &= nP(h(x)=h(y))P(s(x)=s(y))\\
             &\leq n\frac{1}{n}\frac{1}{n^3}\\
             &= \frac{1}{n^3}
\end{align*}
The first equality holds because the hash functions are independent. The next equality uses that $\abs{S}=n$. The inequality uses that the $h$ and $s$ are universal hash functions. So the probability for the hash function and for the signature function have a collision for two distinct keys are less than $1/n^3$
\section*{Ex.2.6}
\subsection*{Multiply-mod-prime hash function where $a=0$ is possible}

\emph{Show it is not universal.}

To be a universal hash function it must be that for all distinct keys $P(h(x)=h(y))\leq 1/m$.

We will give a concrete example where the above probability is not fulfill. Set $p=3$, $m=2$, $x=0$, $y=1$. Then the possible values for $a,b\in\{0,1,2\}$. The table below shows the hash value for $x$ and $y$ with the value set above.

\begin{center}
\begin{tabular}{l|cc|r}
              & $h(x)$ & $h(y)$ & collision\\
\hline
 $(a,b)=(0,0)$& 0 & 0 & true\\
 $(a,b)=(0,1)$& 1 & 1 & true\\
 $(a,b)=(0,2)$& 0 & 0 & true\\
 $(a,b)=(1,0)$& 0 & 1 & false\\
 $(a,b)=(1,1)$& 1 & 0 & false\\
 $(a,b)=(1,2)$& 0 & 0 & true\\
 $(a,b)=(2,0)$& 0 & 0 & true\\
 $(a,b)=(2,1)$& 1 & 0 & false\\
 $(a,b)=(2,2)$& 0 & 1 & false
\end{tabular}	
\end{center}

It can be seen that for 5 out of the 9 possibilities, there is a collsion. Hence the probaility for collision is
$$
P(h(0)=h(1)) = \frac{5}{9} > \frac{1}{2} = \frac{1}{m}
$$
So we have found an instance there this hash function is not universal.

\emph{Prove that it is always 2-universal}

First note that for $a=0$ $P(h_{0,b}(x)=h_{0,b}(y)) = 1$ for all values of $b, x, y$. The probability $P(a=0)=1/p$. To prove 2-universal we see that
\begin{align*}
P(h_{a,b}(x) = h_{a,b}(y)) &= P([(a=0)\cap(h_{a,b}(x) = h_{a,b}(y))]\cup [(a>0)\cap(h_{a,b}(x) = h_{a,b}(y))])\\
&= P((a=0)\cap(h_{0,b}(x) = h_{0,b}(y))) + P((a>0)\cap(h_{a,b}(x) = h_{a,b}(y)))\\
&= \frac{1}{p} + P((a>0)\cap(h_{a,b}(x) = h_{a,b}(y))) \\
&< \frac{1}{p} + \frac{1}{m} \\
&< \frac{1}{m} + \frac{1}{m} \\
&= \frac{2}{m}
\end{align*}
Hence we have shown that the hash function is 2-universal.
\section*{Ex.3.1}
\subsection*{what is it to be $k$-independence}
Define $k$-independence as: Given $k$ distinct keys $x_1,\ldots,x_k\in{U}$ and $k$ non--distinct hash values $q_1,\ldots,q_k\in[m]$, then it must hold that 
$$
P(\bigcap_{i=1}^k (h(x_i)=q_i)) = \frac{1}{m^k}.
$$

To show 3-independece, set $k=3$ in above and 
$$
P(h(x_1)=q_1 \land h(x_2)=q_2 \land h(x_3)=q_3) = \frac{1}{m^3}.
$$
\section*{Ex.3.2}
\subsection*{what is the upper bound}

As $h$ is strongly $c$-universal they keys hash independently, hence

\begin{align*}
P(h(x)=q \land h(y)=r) &= P(h(x)=q \cap h(y)=r) = P(h(x)=q) P(h(y)=r) \\
                       &\leq \frac{c}{m} \frac{c}{m} = \left(\frac{c}{m}\right)^2
\end{align*}

So the upper bound is $c^2/m^2$
\section*{Ex.3.3}
\subsection*{Argue that if $h$ is strongly universal, then it is also universal.}

To be $c$-universal it must hold that $P(h(x) = h(y))\leq c/m$. To be strongly $c$-universal it must hold that $P(h(x) = h(y))\leq c^2/m^2$. Hence if we can show 
$$\left(\frac{c}{m}\right)^2\leq\frac{c}{m}$$ 
the statement holds. As $\frac{c}{m}$ is the probability of a collision, the value must be in the interval [0,1]. Hence $\left(\frac{c}{m}\right)^2 < \frac{c}{m}$.
\section*{Ex.3.4}
\subsection*{Is multiply--Shift strongly universal}

We want to show that it is not by showing it does not hold for $c=1$. We do this by showing that the distinct keys does not hash independently.

So assume that it is strongly universal, then by definition  for any distinct keys $x,y$ and any non-distinct hash values $q,r$ the probability $P((h_b(x) = q) \cap (h_b(y) = r)) = \frac{1}{m^2}>0$. As strongly universality holds for any key, we pick $x=0, q=1$ and $y\neq 0$. Furthermore; it is easy to see that $h_b(0)=0$ for all values of $b$. As we assumed $h$ is strongly universal, then the keys hash independently and
$$
P((h_b(0) = 1) \cap (h_b(y) = r)) = P(h_b(0) = 1) P(h_b(y) = r) = 0 P(h_b(y) = r) = 0.
$$

Hence by assuming string universality we have now shown
$$
0< \frac{1}{m^2} = P((h_b(0) = 1) \cap (h_b(y) = r)) = 0 P(h_b(y) = r) = 0.
$$

which is a contradiction.

As our only assumption is that $h$ is strongly universal. That assumption must be false. Hence we have proven that $h$ is \emph{not} strongly universal.
\section*{Ex.3.5}
\subsection*{Estimate the size of the symmetric difference}
The symmetric difference is $(B\backslash C)\cup (C \backslash B)$.

We have that 
$$
(B\backslash C)\cup (C \backslash B) = (B \cup C)\backslash (C \cap B)
$$
Hence
\begin{align*}
S_{h,t}((B\backslash C)\cup (C \backslash B)) 
  &= S_{h,t}((B \cup C)\backslash (C \cap B))\\
  &= ((S_{h,t}(B) \cup S_{h,t}(C))\backslash (S_{h,t}(C) \cap S_{h,t}(B)))
\end{align*}
Hence the estimate would be
$$
[\left|S_{h,t}(B) \cup S_{h,t}(C)\right| - \left|S_{h,t}(C) \cap S_{h,t}(B)\right|]\frac{m}{t}
$$
\section*{Ex.3.6}
\subsection*{Give upper bound for probability}

We are given $\abs{A}=10^8, p=1/100, \mu=10^6$. 

First note that $\sqrt{\mu} = 1,000$, we want to use Lemma~3.2 to bound $\abs*{X-\mu}\geq 10,000$. Hence we find that $q = 10$ as $q\mu= 10,000$. So
$$
P(\abs*{X-10^6}\geq 10^4)\leq \frac{1}{10^2} = \frac{1}{100}
$$
So the upper bound is 1\%.
\section*{Exercise 20.3-1}
\subsection*{Modify vEB trees to support duplicate keys}

Instead of the tree holding bits, it could instead hold integers. These integers would then act as counters for the number of instances of each key.
\\
Furthermore, to support this change, the element stored in $min$ should appear in the clusters (like $max$), and hold an integer like the rest of the tree.
\\
When inserting and deleting in the tree, the counter should then be updated, by adding to or subtracting from the counter.
\section*{Exercise 20.3-2}
\subsection*{Modify vEB trees to support keys that have associated satellite data}

For both of the keys in the vEB(2) trees (meaning leaves), which are not summaries, we define an additional field, holding the associated satellite data.
\\
Furthermore, we also define this new field for every $min$ in the (entire) tree, that is not a summary, since the elements stored in $min$ does not appear in the clusters.
\\
This way, the new field holds the associated satellite data for every key in the entire tree, but it will not affect running time, since the satellite data are found together with the keys.
\section*{Exercise 20.3-3}
\subsection*{Write pseudocode for a procedure that creates an empty van Emde Boas tree}

\begin{algorithm}
\caption*{vEB-TREE-EMPTY($V,U$)}\label{euclid}
\begin{algorithmic}[1]
\State $V.min$ = NIL
\State $V.max$ = NIL
\State $V.u = U$
\If {$u \neq 2$}
\State $V.summary =$ vEB-TREE-EMPTY$(V, \sqrt[\uparrow]{u})$
\EndIf
\State $V.cluster = Array(\sqrt[\uparrow]{u})$
\For{$i$ in $V.cluster$}
\State $V.cluster[i] = $vEB-TREE-EMPTY$(V, \sqrt[\downarrow]{u})$
\EndFor
\end{algorithmic}
\end{algorithm}

Lines 1-2 sets $min$ and $max$ to NIL in the current tree $V$. In line 3, the universe size $u$ of the tree is set from the input $U$.
\\
Line 4 and 5 creates a new summary of size $\sqrt[\uparrow]{u}$, by making a recursive call, unless $u$ equals the base size 2.
\\
Line 6-8 creates clusters, by creating arrays with the positions $0,...,\sqrt[\uparrow]{u}$, and in each of these positions a recursive call is made with the universe size $\sqrt[\downarrow]{u}$.
\section*{Exercise 20.3-4}
\subsection*{What happens if you call vEB-TREE-INSERT with an element, that is already in the vEB tree?}

The procedure will terminate without inserting anything:
\\
Line 1-2 is not possible, since there is at least one element in the tree. We know this, since $x$ is a duplicate of an element in the tree.
\\
Line 3-4 is not possible either, since $x$ is a duplicate of an element in the tree, so the lowest possible value for $x$ is $x=V.min$
\\
Line 6-8 is not possible either, so eventually we will hit line 9, where the recursive call brings us to a smaller cluster. This will keep happening until we hit a base-case vEB tree, meaning we no longer fulfill line 5, and moves on to line 10-11.
\\
Since line 10-11 is not possible either (again, $x$ is a duplicate of an element in the tree, so the highest possible value for $x$ is $x=V.max$), so the procedure terminates without doing anything.

\subsection*{What happens if you call vEB-TREE-DELETE with an element, that is not in the vEB tree?}

A wrong value will be deleted:
\\
Line 1-3 simply checks if the tree has only one element, and if so, deletes it, regardless of the value of $x$.
\\
Line 4-8 checks if we are in a base-case vEB tree, and if so, updates $V.min$ and $V.max$, based on the value of $x$, which is assumed to be either 0 or 1 (the only two possible values in the base-case vEB tree). Since $x$ is non of these values, this update will end up deleting $V.max$ and leaving $V.min$ unchanged.
\\
Line 9-12 is not possible, since we know, that $x$ is not in the tree, and so we will hit line 12, where the recursive call brings us to a smaller cluster. This will keep happening until we hit a base-case vEB tree, which brings us to line 4-8, which ends up deleting $V.max.$
\\
Line 14-22 is not possible to reach.

\subsection*{Explain why the procedures exhibit the behavior that they do}

Recursive calls will always bring us down to a smaller tree, until we hit a base case.

\subsection*{Show how to modify vEB trees and their operations so that we can check in constant time whether an element is present}

We introduce an extra array, which has $u$ elements, that are bits. For each element in the tree, the value of the corresponding entry in the array represents, whether the element is present (1) or not (0).
\section*{Exercise 20.3-5}
\subsection*{Running times for the modified operations}

The recursive procedures that implement the vEB tree operations have running times characterized by the recurrence
$$
T(u) \leq T(u^{1/k}) + T(u^{1-1/k}) + O(1),
$$
where the first part is due to the number of clusters, and the second part is due to the size of the clusters, which are enumerated.
\\
By letting $m=\lg u$, such that $u=2^m$, then
$$
T(2^m) \leq T(2^{m/k}) + T(u^{m-m/k}) + O(1)
$$
We then rename $S(m)=T(2^m)$, which results in
$$
S(m) \leq S(m/k) + S(m-m/k) + O(1) = S(m/k) + S\left(\frac{m(k-1)}{k}\right) + O(1)
$$
For $k>2$, the second part will dominate, so in terms of running time, we can "ignore" the first term. In this case, case 2 of the master theorem applies, and $S(m)$ has the solution $S(m)=O(\lg m)$.
By changing back, we can conclude, that
$$
T(u) = T(2^m) = S(m) = O(\lg m) = O(\lg\lg u)
$$
\section*{Exercise 20.3-6}
\subsection*{The smallest number of operations $n$ for which the amortized time of each operation in a vEB tree is $O(\lg\lg u)$}

The amortized time of each operation are calculated, by taking the sum of the time for each operation, and dividing this by the number of all operations.
\\
Now, creating the entire universe takes $\Theta (u)$ time, which must account for all operations, so we can express the amortized time of each operation by
$$
\lg\lg u = \frac{u}{n}
$$
giving us $n = \frac{u}{\lg\lg u}$. Hence, we need at least $\frac{u}{\lg\lg u}$ operations.
\end{document}